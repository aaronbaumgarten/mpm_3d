\chapter{Installation}
\section{Downloading the Code}
\texttt{mpm\_3d} is maintained in a git repository here: \url{http://jabroni.mit.edu/gitlab/asbaumgarten/mpm\_3d.git}. After setting up a git account at \url{http://jabroni.mit.edu} and following the instructions to add your SSH key (\url{http://jabroni.mit.edu/gitlab/help/gitlab-basics/README.md}), you can clone the git repo on the command line:

\texttt{\$ git clone git@jabroni.mit.edu:asbaumgarten/mpm\_3d.git}


\section{Code Dependencies}
\texttt{mpm\_3d} is built with CMake 3.2.2 using gcc 5.2.1,
though earlier versions may be supported. \texttt{mpm\_3d} also requires the Eigen linear algebra library for C++. CMake and Eigen can be installed on Linux using the following commands:

\texttt{\$ sudo apt-get install cmake}

\texttt{\$ sude apt-get install libeigen3-dev}

If Eigen is installed somewhere other than \texttt{/usr/include/eigen3}, you will need to edit the CMakeLists.txt file in the main project folder. In particular, you will need to add:

\texttt{include\_directories(PATH-TO-EIGEN3)}


\section{Building the Code}
\texttt{mpm\_3d} can be built from the command line using CMake. After cloning the git repository, navigate to the main project directory and make a build directory:

\texttt{\$ cd mpm\_v3}

\texttt{\$ mkdir build}

To build the code, simply call \texttt{cmake} from the build directory, then \texttt{make}:

\texttt{\$ cd build}

\texttt{\$ cmake ..}

\texttt{\$ make}

If all goes according to plan, \texttt{mpm\_3d} should now be installed!